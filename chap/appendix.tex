%% ----------------------------------------------------------------------------
% BIWI SA/MA thesis template
%
% Created 09/29/2006 by Andreas Ess
% Extended 13/02/2009 by Jan Lesniak - jlesniak@vision.ee.ethz.ch
%% ----------------------------------------------------------------------------
\chapter{The First Appendix}

In the appendix, list the following material:

\begin{itemize}
    \item Data (evaluation tables, graphs etc.)
    \item Program code
    \item Further material
\end{itemize}

\chapter{Extended Derivations}
\section{Forward Process Closed-Form}
\label{app:forward}
Starting with transition distributions
\begin{equation}
    q(\bm{x}_t|\bm{x}_{t-1}) = \mathcal{N}(\sqrt{1-\beta_t} \bm{x}_{t-1}, \beta_t I)
\end{equation}
the reparameterization $\alpha = 1 - \beta$ is introduced
\begin{equation}
    q(\bm{x}_t|\bm{x}_{t-1}) = \mathcal{N}(\sqrt{\alpha_t} \bm{x}_{t-1}, (1-\alpha) I)
\end{equation}
which can also be formulated as
\begin{equation}
    q(\bm{x}_t|\bm{x}_{t-1}) = \sqrt{\alpha_t}\bm{x}_{t-1} + \sqrt{1-\alpha_t}\mathcal{N}(\bm{0}, \bm{I}).
\end{equation}
For coherent indexing it is beneficial to switch to notation using random variables
\begin{equation}
    \bm{x}_{t} = \sqrt{\alpha_t}\bm{x}_{t-1} + \sqrt{1-\alpha_t}\bm{\epsilon_{t-1}}
    \label{eq:forward_randomvar}
\end{equation}
where $\bm{\epsilon_{t-1}} \sim \mathcal{N}(\bm{0}, \bm{I})$ and the earlier $\bm{x}_t$ can be recursively inserted into the formula. Recalling that the sum $Z = X + Y$ of two normally distributed random variables $X \sim \mathcal{N}(\mu_X, \sigma_Y^2)$ and $Y \sim \mathcal{N}(\mu_Y, \sigma_Y^2)$ is again normally distributed according to $Z \sim \mathcal{N}(\mu_X + \mu_Y, \sigma_X^2 + \sigma_Y^2)$
\begin{align}
    x_t & = \sqrt{\alpha_t} \left( \sqrt{\alpha_{t-1}} \bm{x}_{t-2} + \sqrt{1-\alpha_{t-1}}\bm{\epsilon}_{t-2} \right) + \sqrt{1-\alpha_{t}} \bm{\epsilon}_{t-1} \\
        & = \sqrt{\alpha_{t}\alpha_{t-1}} \bm{x}_{t-2} + \sqrt{\alpha_{t}(1-\alpha_{t-1})} \bm{\epsilon}_{t-2} + \sqrt{1-\alpha_{t}} \bm{\epsilon}_{t-1}         \\
        & = \sqrt{\alpha_{t}\alpha_{t-1}} \bm{x}_{t-2} + \sqrt{\alpha_{t}(1-\alpha_{t-1}) + (1-\alpha_{t})} \bm{\bar{\epsilon}}_{t-2}
\end{align}
where $\bm{\bar{\epsilon}}_{t-2}$ is the sum of the random variables up to $t-2$ (again Gaussian). The second term can of course be simplified to
\begin{equation}
    \bm{x}_t = \sqrt{\alpha_{t}\alpha_{t-1}} \bm{x}_{t-2} + \sqrt{1-\alpha_t\alpha_{t-1}} \bm{\bar{\epsilon}}_{t-2}
\end{equation}
which is exactly the same form as in Eq.~\ref{eq:forward_randomvar}. Therefore the final form is
\begin{equation}
    \bm{x}_t = \sqrt{\bar{\alpha}_{t}} \bm{x}_{t-2} + \sqrt{1-\bar{\alpha}_{t}} \bm{\bar{\epsilon}}_{t-2}
\end{equation}
with $\bar{\alpha_t} = \prod_{s=1}^{t}\alpha_s$ as before.