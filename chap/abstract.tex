%% ----------------------------------------------------------------------------
% BIWI SA/MA thesis template
%
% Created 09/29/2006 by Andreas Ess
% Extended 13/02/2009 by Jan Lesniak - jlesniak@vision.ee.ethz.ch
%% ----------------------------------------------------------------------------

\noindent Unconditional DDPMs have the potential to be powerful priors for the reconstruction of undersampled MRI and this work explores several ways of conditioning DDPMs for this task. Prior work on image inpainting and image-to-image translation introduced a way of conditioning DDPMs by injecting known information into the latent representations, but this technique failed to translate to the task of MRI reconstruction. In order to identify issues with this conditioning method, variability in samples from DDPMs was studied, both in image and frequency space, in order to determine feature hierachies over the reverse diffusion process. The experiments did not succeed in alleviating the issues, but the interpretation of the DDPM as a score-based model offered a way of integrating DDPM priors into known iterative reconstruction schemes and produced competitive reconstruction results for high undersampling factors. The results with \textit{loss guidance} were even improved by optimizing over more iterations, which results in resampling the DDPM.