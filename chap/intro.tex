%% ----------------------------------------------------------------------------
% BIWI SA/MA thesis template
%
% Created 09/29/2006 by Andreas Ess
% Extended 13/02/2009 by Jan Lesniak - jlesniak@vision.ee.ethz.ch
%% ----------------------------------------------------------------------------


\chapter{Introduction}
\section{Background \& Relevance}
MRI (magnetic resonance imaging) is a medical imaging modality that allows acquiring slices of the body, which is an invaluable non-invasive procedure in medical diagnostics. In contrast to the widely used CT (computed tomography) it does not rely on harmful ionizing radiation and it offers much better soft tissue contrast. This is enhanced by the large flexibility of the acquisition protocols, which can often be tuned to yield the best contrast between tissues of interest. The biggest difficulty with MRI scans are the long acquisition times, which requires patients to lay still for extended amounts of time, which is especially difficult for children and intellectually disabled patients. Additionally, long acquisition times make scans more expensive and available to a smaller number of patients. A significant part of MRI-related research is therefore occupied by reaching acquisition speedups. Such techniques usually rely on several acquisition coils~\autocite{sodickson1997smash,pruessmann1999sense,griswold2002grappa} and on undersampling of the acquisition space, which is the space of spatial frequencies in the case of MRI. This space corresponds to the 2D Fourier transform of the image space and is usually termed \textit{k-space}, relating to the variable $k$, the wave number or spatial frequency. Undersampling k-space poses a challenging inverse problem that can be solved well by compressed sensing techniques~\autocite{donoho2006compressedsensing,candes2005stable} for small accelerations (undersampling factors), but needs additional information from multiple coils for higher accelerations or has to rely on strong priors.

Generative machine learning for images has made huge progress in the last few years, thanks to the incorporation of neural networks and since generative machine learning is concerned with learning data distributions, it offers a possibility for incorporating such strong priors into inverse problems. Among the most influential architectures of the past few years are variational autoencoders (VAEs), generative adversarial networks (GANs) and diffusion denoising probabilistic models (DDPMs).~\autocite{kingma2013autoencoding,goodfellow2014generative,sohldickstein2015deep,ho2020denoising}

\section{Focus of this Work}
By merging the VAE's strong mode coverage with sample quality comparable to GANs, DDPMs have recently emerged as the most powerful model for modeling image distributions~\autocite{dhariwal2021diffusion} and are therefore used in this work. Using large amounts of high-quality MRI data from various acquisition protocols, the focus of this work is on training strongly generalizing DDPMs and subsequently use them as priors for the reconstruction of undersampled k-space. This means that the model is not conditioned on the reconstruction tass at training time, but at inference time. The advantage of this approach is that the model can be used for a variety of image reconstruction tasks and is not limited to the use case of undersampled MRI. Further, for the case of reconstructing undersampled MRI, the reconstruction is not reliant on a set of undersampling masks, known at training time. Thanks to the high interpretability of DDPMs, they allow for different approaches to this conditioning, which are explored in this work. A further focus lies on the exploration of sampling techniques that might give better reconstruction quality by making use of higher computational resources.
\section{Thesis Organization}
In the first part of the thesis, the theoretical framework behind DDPMs is established and related work is introduced, that successfully managed to condition unconditionally-trained DDPMs. In the second part, the conditioning methods are adapted to fit the task of reconstructing undersampled MRI and further, the used model architectures, training protocols and datasets are introduced. The third part shows the experimental results from model training and model condition, and compares the performance between the different conditioning methods, by evaluating them over different accelerations and sampling strategies.