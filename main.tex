%% ----------------------------------------------------------------------------
% BIWI SA/MA thesis template
%
% Created 09/29/2006 by Andreas Ess
% Extended 13/02/2009 by Jan Lesniak - jlesniak@vision.ee.ethz.ch
% Updated 16/03/2023 by Danda Pani Paudel - paudel@vision.ee.ethz.ch
%% ----------------------------------------------------------------------------
\documentclass[pdftex,11pt,openright,headsepline]{book}
\usepackage[margin=1in,headheight=13.6pt]{geometry}

\usepackage{paralist}		% List environment
\usepackage{color}		% For colored text
\usepackage{times}
\usepackage{amsfonts}		% Additional math fonts
\usepackage{amsmath}		% Math symbols
\usepackage{latexsym}
\usepackage{graphicx}		% For including images
% \usepackage{listings}		% If listings are needed
\usepackage{packages/mydefs}		% Some of our own definitions
% \usepackage{wrapfig}		% To wrap images
% \usepackage{algorithmic}	% Nice algorithm environment
% \usepackage{algorithm}
\usepackage{fancyhdr}		% Produce the nice header
%\usepackage{fullpage} % Use the full page

% ========= CHANGES ==================================
\usepackage[english]{babel}
\usepackage[utf8]{inputenc}
\usepackage[T1]{fontenc}
\usepackage[backend=biber, bibencoding=utf8]{biblatex}
\bibliography{bib/references}
\usepackage{csquotes}
% ====================================================


% Change the appearance of the header. Here \MakeUppercase is hard-coded, so renewing this command allows to elegantly change the header appearance.
\renewcommand{\MakeUppercase}{\scshape}

% Set the headings' appearance in the ``fancy'' pagestyle
\fancyhead{}
\fancyhead[RO, LE]{\leftmark}
\fancyfoot{}
\fancyfoot[RO, LE]{\thepage}

% The first pages shall be empty, even no page numbering


\begin{document}
\pagestyle{empty} % even no page number

\fancypagestyle{plain}{
  \renewcommand{\headrulewidth}{0.0pt}
  \fancyfoot{}
  \fancyhead{}
}

% Title page, modify accordingly
%% ----------------------------------------------------------------------------
% BIWI SA/MA thesis template
%
% Created 09/29/2006 by Andreas Ess
% Extended 13/02/2009 by Jan Lesniak - jlesniak@vision.ee.ethz.ch
% Updated 16/03/2023 by Danda Pani Paudel - paudel@vision.ee.ethz.ch
%% ----------------------------------------------------------------------------

\begin{titlepage}

    \thispagestyle{empty}

    \fancypagestyle{empty}{
        \lhead{\includegraphics[height=1.5cm]{images/ethlogo_black}}
        \renewcommand{\headrulewidth}{0.0pt}
        \rhead{\vspace*{-0.2cm}\includegraphics[height=1.4cm]{images/cvl_logo}}
        \fancyfoot{}
    }



    \vspace*{2cm}
    \begin{center}
        \Huge{\textbf{Conditioning of DDPMs on Accelerated MRI}\\}
        \LARGE{\textbf{}\\[1cm]}
        \vspace{5pt}
        \large{Semester Thesis\\[0.8cm]}
        \LARGE{Lionel Peer\\}
        \normalsize{Department of Information Technology and Electrical Engineering}
    \end{center}

    \begin{center}



        % \begin{center}
        % \begin{tabular}{ll}
        % \multirow{2}{*}{\includegraphics[height=1cm]{images/biwi_logo}} & Computer Vision Laboratory\\
        % & ETH Zurich
        % \end{tabular}
        %  \end{center}

    \end{center}


    \vfill
    \begin{center}
        \begin{tabular}{ll}
            \Large{\textbf{Advisors:}}   & \Large{Georg Brunner \& Emiljo Mëhillaj}                                 \\
            \Large{\textbf{Supervisor:}} & \Large{Prof.~Dr.~Ender Konukoglu}                                        \\
                                         & \small{Computer Vision Laboratory, Group for Biomedical Image Computing} \\
                                         & \small{Department of Information Technology and Electrical Engineering}  \\
        \end{tabular}
    \end{center}

    \begin{center}
        % \today\\
        January 10, 2020
    \end{center}


\end{titlepage}


\cleardoublepage

%% ----------------------------------------------------------------------------
% BIWI SA/MA thesis template
%
% Created 09/29/2006 by Andreas Ess
% Extended 13/02/2009 by Jan Lesniak - jlesniak@vision.ee.ethz.ch
%% ----------------------------------------------------------------------------

\noindent Unconditional DDPMs have the potential to be powerful priors for the reconstruction of undersampled MRI and this work explores several ways of conditioning DDPMs for this task. Prior work on image inpainting and image-to-image translation introduced a way of conditioning DDPMs by injecting known information into the latent representations, but this technique failed to translate to the task of MRI reconstruction. In order to identify issues with this conditioning method, variability in samples from DDPMs was studied, both in image and frequency space, in order to determine feature hierachies over the reverse diffusion process. The experiments did not succeed in alleviating the issues, but the interpretation of the DDPM as a score-based model offered a way of integrating DDPM priors into known iterative reconstruction schemes and produced competitive reconstruction results for high undersampling factors. The results with \textit{loss guidance} were even improved by optimizing over more iterations, which results in resampling the DDPM.

% Input here any acknowledgements
%% ----------------------------------------------------------------------------
% BIWI SA/MA thesis template
%
% Created 09/29/2006 by Andreas Ess
% Extended 13/02/2009 by Jan Lesniak - jlesniak@vision.ee.ethz.ch
%% ----------------------------------------------------------------------------


\newpage

\chapter*{Acknowledgements}
I would like to thank my advisors, Emiljo and Georg, for the support, trust and liberty that I was given over the course of this project. I was able to freely decide the course of this project, and discussions and questions were always received with open arms by the them. Further, I would like to thank Professor Konukoglu for enabling this project in his group and last but not least my friends \& family, who made sure that I would balance work and leisure.
\cleardoublepage
\newpage

% % Chapter-pages etc. use the ``plain'' pagestyle - since we don't want to have a heading at all at chapter-pages, redefine plain accordingly. Don't forget the page number.
\fancypagestyle{plain}{
  \renewcommand{\headrulewidth}{0.0pt}
  \fancyfoot{}
  \fancyfoot[RO, LE]{\thepage}
  \fancyhead{}
}

\pagestyle{fancy}
\pagenumbering{Roman}

% Insert table of contents
\tableofcontents

% Insert list of figures
\listoffigures
\cleardoublepage

% Insert list of tables
\listoftables
\cleardoublepage

\newpage

\pagenumbering{arabic}

%% ----------------------------------------------------------------------------
% Actual text comes here - defer it to other files and use \input{bla.tex}, ..
%% ----------------------------------------------------------------------------
%% ----------------------------------------------------------------------------
% BIWI SA/MA thesis template
%
% Created 09/29/2006 by Andreas Ess
% Extended 13/02/2009 by Jan Lesniak - jlesniak@vision.ee.ethz.ch
%% ----------------------------------------------------------------------------


\chapter{Introduction}
\section{Background \& Relevance}
MRI (magnetic resonance imaging) is a medical imaging modality that allows acquiring slices of the body, which is an invaluable non-invasive procedure in medical diagnostics. In contrast to the widely used CT (computed tomography) it does not rely on ionizing radiation, which is harmful to the cells in large doses, and offers much better soft tissue contrast. As an additional benefit, acquisition protocols are highly flexible and often allow to tuning the contrast to tissues of interest. The biggest difficulty with MRI scans are the long acquisition times that require patients to lay still for extended amounts of time, which is especially difficult for children and intellectually disabled patients. Additionally, long acquisition times make scans more expensive and available to a smaller number of patients. A significant part of MRI-related research is therefore A speedup, without a drop in image quality, can be achieved by the use of several acquisition coils~\autocite{sodickson1997smash,pruessmann1999sense,griswold2002grappa} or by undersampling the acquisition space, which, in the case of MRI, is the space of spatial frequencies. This space corresponds to the 2D Fourier transform of the image space and is usually termed \textit{k-space}, relating to the variable $k$, which is usually assigned to the spatial frequency. Undersampling k-space poses a challenging inverse problem that can be solved well by compressed sensing techniques~\autocite{donoho2006compressedsensing,candes2005stable} for small accelerations (undersampling factors), but usually requires stronger priors for higher accelerations. Since generative machine learning is concerned with learning data distributions it offers a possibility for incorporating such strong priors into inverse problems and generative machine learning for images has made huge progress in the last few years, thanks to the incorporation of neural networks. Visionary works in the domains of variational autoencoders (VAEs), generative adversarial networks (GANs) and diffusion denoising probabilistic models (DDPMs) have opened the door to a variety of models that can learn complicated image distributions and produce samples of photo-realistic quality.~\autocite{kingma2013autoencoding,goodfellow2014generative,sohldickstein2015deep,ho2020denoising}

\section{Focus of this Work}
DDPMs have recently emerged as the most powerful model for modeling image distributions and therefore they are the model used in this work. The focus is to train DDPMs on large amounts of high-quality MRI data from various acquisition protocols and to then use this prior for reconstruction of undersampled k-space. The advantage of this approach is that the type of undersampling does not need to be known at training time, and that a single model could also be used for a variety of other image reconstruction tasks. This means that the model needs to be conditioned on the task of reconstructing undersampled MRI post-training. Thanks to the high interpretability of DDPMs, they allow for several different approaches to this conditioning, which are explored in this work. A further focus lies on the exploration of sampling techniques that might give better reconstruction quality by making use of higher computational resources.
\section{Thesis Organization}
In the first part of the thesis, the theoretical framework behind DDPMs is established and related work is introduced, that successfully managed to condition unconditionally trained DDPMs. In the second part, the introduced conditioning methods are adapted to fit the task of reconstructing undersampled MRI and the used model architectures, training protocols and datasets are introduced. The third part shows the experimental results from model training and compares the performance between the different conditioning methods, by evaluating them over different accelerations and sampling strategies.
%% ----------------------------------------------------------------------------
% BIWI SA/MA thesis template
%
% Created 09/29/2006 by Andreas Ess
% Extended 13/02/2009 by Jan Lesniak - jlesniak@vision.ee.ethz.ch
%% ----------------------------------------------------------------------------
\newpage
\chapter{Related Work}
\section{Latent Variable Models}
Latent variable models are generative models which assume that it is possible to model the true data distribution $p(x)$ as a joint distribution $p(x,z)$, where $x$ and $z$ are multi-variate random vectors.
\begin{equation}
    \label{eq:marginallikelihood}
    p(x) = \int p(x,z)dz = \int p(x|z)p(z)dz
\end{equation}
Many naturally occurring distributions of samples can be imagined to come from a much simpler underlying distribution, which is obscured by the space that they are observed in. This is the main motivation behind latent variable models and in order to understand these models, it is important to define the terms used in the next sections, since they all stem from Bayesian statistics. The Bayesian theorem can be written as
\begin{equation}
    \label{eq:bayestheorem}
    p(z|x) = \frac{p(x|z)p(z)}{p(x)}
\end{equation}
which is a generally applicable formula for any conditional distributions, but in generative modeling and machine learning it is usually assumed that the letter $z$ represents a random vector of a simpler distribution in the latent (unobserved) space, and $x$ is the random vector modeling the complicated distribution in the observed space (the sample space). The four terms in the formula use distinct names:
\begin{description}
    \item[$p(x)$] is called the \textit{evidence} or the \textit{marginal likelihood}. It encompasses the actual observations of the data.
    \item[$p(z)$] is called the \textit{prior}, since it exposes information on $z$ before any conditioning.
    \item[$p(z|x)$] is called the \textit{posterior}. It describes the distribution over $z$ after (\textit{post}) having seen the evidence $x$.
    \item[$p(x|z)$] is called the \textit{likelihood}, since it gives the likelihood of observing an example $x$ when choosing the latent space to be a specific $z$.
\end{description}

\subsection{Variational Autoencoders}
One of the most straightforward examples of a generative model, where the goal is to find such a latent space representation of the training sample distribution, is the Variational Autoencoder (VAE)~\autocite{kingma2013autoencoding}. For the two factors in Eq.~\ref{eq:marginallikelihood}, the VAE uses a simple multivariate distribution as the latent $p_{\theta_z}(z)$ (e.g. a multivariate Gaussian) and a neural network mapping $p_{\theta_{NN}}(x|z)$. Training then includes finding optimal parameters for the parameterized latent distribution and for the neural network mapping, such that sampling $z$ and mapping it to the sample space is almost the same as sampling $x$ directly. When no prior over the parameters $\theta_z, \theta_{NN}$ is considered, this is usually done through an MLE (maximum likelihood estimate) $\hat{\theta} = \argmax_{\theta} p_{\theta}(x)$. While simple latent variable models can be optimized through differentiation, or iterative algorithms such as EM (expectation maximization) and gradient descent, these algorithms usually don't work for complicated multi-modal distributions (as parameterized by neural networks), since the integral in Eq.~\ref{eq:marginallikelihood} has no closed form solution and is also difficult or costly to estimate. Therefore it would be preferred to use a parameterization instead that also uses an estimation of the posterior $p_{\theta}(z|x) \approx p(z|x)$.
\begin{equation}
    \label{eq:likelihoodvae}
    p_{\theta}(x) = \int p_{\theta_{NN}}(x|z) p_{\theta_z}(z) dz
\end{equation}
\begin{figure}[h]
    \centering
    \includegraphics[width=.5\textwidth]{images/vae.png}
    \caption[VAE schematic]{VAE schematic: $p(x)$ is approximated through a latent variable model where posterior and likelihood are modeled with neural networks and the prior on the latent variable is modeled through a simple parameterized distribution (often Gaussian). The hope is, that after training, sampling from $p(z)$ and passing it through the neural network $p_{\theta_{NN}}(x|z)$, is the same as sampling from $p(x)$.}
    \label{fig:vae}
\end{figure}
Figure
The name of the VAE stems from the Autoencoder, a neural network that learns to recreate its output through an encoder with a bottleneck and a decoder, thereby learning a compressed representation of the data at the bottleneck and Fig.~\ref{fig:vae} illustrates the connection.~\autocite{https://doi.org/10.1002/aic.690370209} Autoencoders bear similarity to other dimension reduction methods like Principal Component Analysis (PCA) and therefore were first published under the name \textit{Nonlinear principal component analysis}.
\subsection{KL Divergence and Variational Lower Bound}
In VAEs, the encoder $p_{\theta}(z|x)$ needs to approximate the true posterior $p(z|x)$ and sampled data should look like it was sampled from $p(x)$. This requires a measure that can compare the similiarity between two probability distributions. One such heavily used measure is the KL (Kullback-Leibler) divergence, formulated for the posterior and its approximation as
\begin{equation}
    \label{eq:kldivergence}
    KL\left[p_{\theta}(z|x) || p(z|x)\right] = \int \log \frac{p_{\theta}(z|x)}{p(z|x)} p_{\theta}(z|x) dz = \mathbb{E}_{z\sim p_{\theta}(z|x)}\left[\log \frac{p_{\theta}(z|x)}{p(z|x)}\right].
\end{equation}
The KL divergence has the properties of being strictly non-negative and only being 0 if the two distributions are equal, but the proofs of those properties are omitted in this work.

The problem with the KL divergence in Eq.~\ref{eq:kldivergence} is that the true posterior is unknown. We will therefore introduce a loss function called ELBO (evidence lower bound) or VLB (variational lower bound) that automatically makes sure that the KL divergence between the parameterized posterior and the true posterior is minimized, without knowing $p(z|x)$. For understanding the ELBO, it is important to note that the marginal log-likelihood can be written as follows (derivation in the appendix):
\begin{align}
    \log p_{\theta}(x) & = \mathbb{E}_{z\sim p_{\theta_{NN}}(z|x)}\left[\log \frac{p_{\theta_{NN}}(x|z) p_{\theta_z}(z)}{p_{\theta_{NN}}(z|x)}\right] + KL\left[p_{\theta_{NN}}(z|x)||p(z|x)\right]
\end{align}
From the properties of the KL divergence we know that the second term on the right hand side is strictly non-negative, this means that the first term on the right hand side offers a lower bound to the log-likelihood of the data and the difference between that first term and the log-likelihood of the data is exactly the KL divergence that we wanted to minimize in Eq.~\ref{eq:kldivergence}. The relationship is also illustrated in Fig.~\ref{fig:elbo}.
\begin{equation}
    \label{eq:elbo}
    \log p_{\theta}(x) \geq \mathbb{E}_{z\sim p_{\theta_{NN}}(z|x)}\left[\log\frac{p_{\theta_{NN}}(x|z) p_{\theta_z}(z)}{p_{\theta_{NN}}(z|x)}\right]
\end{equation}
\begin{figure}[h]
    \centering
    \includegraphics[width=.5\textwidth]{images/elbo.png}
    \caption[Illustration of the ELBO/VLB]{Illustration of the ELBO/VLB: The ELBO term gives a lower bound to the log-likelihood since the KL divergence is strictly non-negative. By maximizing the ELBO term, the KL divergence term is implicitly minimized and ELBO term converges towards the log-likelihood.}
    \label{fig:elbo}
\end{figure}
This means that, if we could maximize the term ELBO term from Eq.~\ref{eq:elbo} it would not only approach the log-likelihood, but simultaneously make sure that the estimated posterior converges to the true posterior. Luckily, for the parameterization of the VAE, the ELBO term can be split into two interpretable parts for optimization.
\begin{align}
    \mathbb{E}_{z\sim p_{\theta_{NN}}(z|x)}\left[\log\frac{p_{\theta_{NN}}(x|z) p_{\theta_z}(z)}{p_{\theta_{NN}}(z|x)}\right] & = \mathbb{E}_{z\sim p_{\theta_{NN}}(z|x)}\left[\log p_{\theta_{NN}}(x|z)\right] - \mathbb{E}_{z\sim p_{\theta_{NN}}(x|z)}\left[\log \frac{p_{\theta_{z}}(z)}{p_{\theta_{NN}}(z|x)}\right]                                        \\
                                                                                                                              & = \underbrace{\mathbb{E}_{z\sim p_{\theta_{NN}}(z|x)}\left[\log p_{\theta_{NN}}(x|z)\right]}_{\text{reasonable reconstruction}} - \underbrace{KL \left[p_{\theta_{NN}}(z|x)||p_{\theta_{z}}(z)\right]}_{\text{correct encoding}}
\end{align}
Maximizing the first part makes sure that the decoder reconstructs reasonable samples from the latent distribution, minimizing the second makes sure that the encoder transforms the training data into our chosen prior over the latents $z$ (usually Gaussian, as mentioned before). The reconstruction term is trivially maximized by minimizing some loss between input and output and if the prior $p(z)$ is chosen to be a Gaussian $p_{\theta_{z}}(z)$, then the KL divergence has a closed form, the derivation of which is omitted.~\autocite{mreasykldivergence}
\begin{equation}
    D_{KL}(p||q) = \frac{1}{2}\left[\log\frac{|\Sigma_q|}{|\Sigma_p|} - k + (\boldsymbol{\mu_p}-\boldsymbol{\mu_q})^T\Sigma_q^{-1}(\boldsymbol{\mu_p}-\boldsymbol{\mu_q}) + tr\left\{\Sigma_q^{-1}\Sigma_p\right\}\right]
\end{equation}

In the next section, the DDPM (diffusion denoising probabilistic model) will be introduced, which is the model architecture used throughout this work. As will be clear shortly, DDPMs can be viewed as a chained VAE that uses a sequence of latent spaces. This is an arguably easier learning problem, since the neural network does not have to map directly from noise to samples, but can do so in an iterative process over many steps.

\section{Diffusion Denoising Probabilistic Models}
Diffusion Denoising Probabilistic Models (DDPMs or Diffusion Models) are a generative model that learn the distribution of images in a training set. During training, sample images are gradually destroyed by adding noise over many iterations and a neural network is trained, such that these steps can be inverted.

As the name suggests, image content is diffused in timesteps, therefore we use the random variable $\bm{x}_0$ to represent our original training images, $\bm{x}_t$ for (partially noisy) images at an intermediate timestep and $\bm{x}_T$ for images at the end of the process where all information has been destroyed, and the distribution $q(\bm{x}_T)$ largely follows an isotropic Gaussian distribution.

The goal is to train a network that creates a less noisy image $\bm{x}_{t-1}$ from $\bm{x}_t$. If this is achieved over the whole training distribution, then sampling new $\bm{x}_T$ and passing and iteratively denoising it, should be the same as sampling $q(\bm{x}_0)$ directly.

\subsection{Forward Diffusion Process}
In order to derive a training objective it is important to understand the workings of the \textit{forward diffusion process}. During this process, i.i.d (independent and identically distributed) Gaussian noise is applied to the image over many discrete timesteps. A \textit{variance schedule} defines the means and variances ($\sqrt{1-\beta}$ and $\beta$) of the added noise at every timestep.~\autocite{ho2020denoising} The whole process can be expressed as a Markov chain (depicted in Fig.~\ref{fig:forward_diffusion}), with the factorization
\begin{align}
    \label{eq:forwardprocess}
    q(\bm{x}_{0:T})            & = q(\bm{x}_0) \prod_{t=1}^{T} q(\bm{x}_{t}|\bm{x}_{t-1}) &  & \text{(joint distribution)}       \\
    q(\bm{x}_{0:T}|\bm{x}_{0}) & = \prod_{t=1}^{T} q(\bm{x}_{t}|\bm{x}_{t-1})             &  & \text{(forwarding single sample)}
\end{align}
where the transition distributions $q(\bm{x}_t|\bm{x}_{t-1}) = \mathcal{N}(\sqrt{1-\beta_t} \bm{x}_{t-1}, \beta_t I)$ and we used the shorthand notation $\bm{x}_{0:T} = \bm{x}_{0},\dots,\bm{x}_{T}$. An example of iterative destruction of an image by this process is shown in Fig.~\ref{fig:forward_naoshima}.

\begin{figure}[h]
    \centering
    \includegraphics[width=.5\textwidth]{images/forward_diffusion.png}
    \caption[Markov Chain Interpretation of Forward Diffusion Process]{Markov Chain Interpretation of Forward Diffusion Process: An image is iteratively destroyed by adding normally distributed noise,
        according to a schedule. This represents a Markov process with the transition probability $q(\bm{x}_t|\bm{x}_{t-1})$.}
    \label{fig:forward_diffusion}
\end{figure}

\begin{figure}[h]
    \centering
    \includegraphics[width=\textwidth]{images/forward_naoshima.png}
    \caption[Example of Iterative Image Destruction through Forward Diffusion Process]{Example of Iterative Image Destruction through Forward Diffusion Process:
        The indices give the time step in the iterative destruction process, where $\beta$ was created according to a linear noise variance schedule (5000 steps from in the 0.001 to 0.02 range and picture resolution of 4016$\times$6016 pixels).}
    \label{fig:forward_naoshima}
\end{figure}

Gladly it is not necessary to sample noise again and again in order to arrive at $\bm{x}_t$, since Ho et al. derived a closed-form solution to the sampling procedure.~\autocite{ho2020denoising} For this, the variance schedule is first reparameterized as $1-\beta = \alpha$
\begin{equation}
    q(\bm{x}_t | \bm{x}_{t-1}) = \mathcal{N}(\sqrt{\alpha_t} \bm{x}_{t-1}, (1-\alpha_t) \bm{I})
    \label{eq:forward_alpha}
\end{equation}
and the closed-form solution for $q(\bm{x}_t|\bm{x}_0)$ is derived by introducing the cumulative product $\bar{\alpha}_t = \prod_{s=1}^{t}\alpha_s$ as
\begin{equation}
    q(\bm{x}_t|\bm{x}_0) = \mathcal{N}(\sqrt{\bar{\alpha}_t}\bm{x}_0, (1-\bar{\alpha}_t)\bm{I})
    \label{eq:forward_alphadash}
\end{equation}
The derivation that leads from Eq.~\ref{eq:forward_alpha} to Eq.~\ref{eq:forward_alphadash} is left to appendix~\ref{app:forward}.

A choice of $\bar{\alpha}_t \in [0,1]$ in above parameterizaiton ensures that the variance does not explode in the process, but that the SNR (signal-to-noise-ratio) still goes to 0 by gradually attenuating the means, corresponding to the original image. Thanks to the reparameterization with $\bar{\alpha}_t$, the forward process is also not restricted anymore to discrete timesteps, but a continuous schedule can be used.~\autocite{kingma2023variational,song2021scorebased}

The process of information destruction is dependent on the chosen variance schedule, the number of steps and the image size. Beyond the most simple case -- a constant variance over time -- Ho et al. opted for the second most simple option, a linear schedule, where the variance $\beta_t$ grows linearly in $t$.~\autocite{ho2020denoising} Nichol et al. later found that a cosine-based schedule gives better results on lower resolution images, since it does not destruct information quite as quickly, making it more informative in the last few timesteps. They also mention that their cosine schedule is purely based on intuition and they expect similar functions to perform equally well.~\autocite{nichol2021improved}  Own experiments exploring above mentioned parameters are explained in~\ref{sec:forward_diff_experiments} and plots of the two different variance schedules are visible in Fig.~\ref{fig:alphadash}.

\subsection{Reverse Diffusion Process}
As mentioned before DDPMs can be viewed as latent space models in a similar way that Generative Adversarial Nets or Variational Autoencoders can.~\autocite{goodfellow2014generative,kingma2013autoencoding} In DDPMs the reverse process is essentially again a Markov chain and can therefore again be factorized as
\begin{equation}
    \label{eq:reverseprocess}
    q(\bm{x}_{0:T}) = q(\bm{x}_T) \prod_{t=T}^{1} q(\bm{x}_{t-1}|\bm{x}_{t})
\end{equation}
where we start from $\bm{x}_T\sim\mathcal{N}(0,\bm{I})$. This means that the network does not learn to approximate the full inversion, but rather just the transition probabilities $q(\bm{x}_{t-1}|\bm{x}_{t})$ in the chain, which are transitions between several intermediate latent distributions. During training, we will need to condition the inversion on a training sample, where the Markov properties of the reverse process will no longer hold. In the appendix, it is shown that the inversion is also Gaussian, we therefore train a neural network to approximate
\begin{equation}
    \label{eq:reverseapprox}
    q(\bm{x}_{t-1} | \bm{x}_t) \approx p_{\theta}(\bm{x}_{t-1} | \bm{x}_t) = \mathcal{N}(\bm{\mu}_{\theta}(\bm{x}_t, t),\bm{\Sigma}_{\theta}(\bm{x}_t, t)).
\end{equation}

\subsection{Loss Functions}
The combination of forward $q(\bm{x}_T|\bm{x}_0)$ and reverse process $q(\bm{x}_0|\bm{x}_T)$ can be viewed as a chain of VAEs and we can again formulate a variational lower bound objective like before. The lengthy derivation of the ELBO for the DDPM is omitted in this work, but can be looked up in the Calvin Luo's work.~\autocite{luo2022understanding} The final form is similar to the one from the VAE, with a reconstruction term and a prior matching term, but with additional terms that match the intermediate latents.
\begin{align}
    \log p_{\theta}(x) & \geq \underbrace{\mathbb{E}_{q(x_1|x_0)} \left[ \log p_{\theta}(x_0|x_1) \right]}_{\text{reasonable reconstruction}}          \\
                       & - \underbrace{KL \left[ q(x_T|x_0) || p(x_t) \right]}_{\text{correct encoding}}                                               \\
                       & - \sum_{t=2}^{T} \underbrace{KL \left[ q(x_{t-1}|x_{t},x_0) || p_{\theta}(x_{t-1}|x_{t}) \right]}_{\text{denoising matching}}
\end{align}
The term $q(x_{t-1}|x_{t},x_0)$ is the true reverse process, conditioned on a single sample. This term comes to be when substituting the posterior transitions $p(x_t|x_{t-1})$ with $p(x_t|x_{t-1}, x_0)$, which is allowed since the Markov property states that $x_t$ only depends on $x_{t-1}$. The natural choice for the denoising matching term would be $KL\left[ q(x_t|x_{t-1}) || p_{\theta}(x_t|x_{t+1}) \right]$, but this has higher variance and is therefore harder to estimate.~\autocite{ho2020denoising} Due to the DDPM usually having 1000 or more timesteps, the ELBO is dominated by the third term. For this reason the first term is usually not used during optimization, since it can only be estimated using Monte Carlo sampling. While it is not used for optimization, it can be useful for evaluating the performance of a trained model. The second term is parameter-free therefore also not used for optimization. It should anyway be zero if the parameterization of the forward process is correct, which means that forward diffused samples get close to our chosen latent prior $p(x_T) = \mathcal{N}(0,\bm{I})$. As mentioned before, $p_{\theta}(x_{t-1}|x_t)$ is also Gaussian and since it was decided to fix the variances of the transitions to a fixed schedule, the variances of the inversion are often fixed as well and only the means are learned. When looking at the formula for the KL divergence between two Gaussians (Eq.~\ref{eq:kldivergence}) with fixed diagonal covariance matrices, one can derive that it reduces to a mean squared error between the distributional means.~\autocite{luo2022understanding}
\begin{equation}
    \hat{\theta} = \argmin_{\theta} KL \left[ q(x_{t-1}|x_{t},x_0) || p_{\theta}(x_{t-1}|x_{t}) \right] = \argmin_{\theta} \frac{1}{2\beta(t)^2} \left[ || \mu_{\theta} - \mu_{q} ||_2^2 \right]
\end{equation}
Ho et al.~\autocite{ho2020denoising} found that it works best, if the network is trained to predict the noise in the image directly and the means are then found through reparameterization
\begin{equation}
    \mu_{\theta}(x_t,t) = \frac{1}{\sqrt{\alpha_t}}x_t - \frac{1-\alpha_t}{\sqrt{1-\bar{alpha}_t}\sqrt{\alpha_t}}\hat{\epsilon}_{\theta}(x_t,t)
\end{equation}
which transforms the loss from before into
\begin{equation}
    \hat{\theta} = \argmin_{\theta}\frac{1}{2\beta(t)^2} \frac{(1-\alpha_t)^2}{(1-\bar{alpha}_t)\alpha_t} \left[ || \epsilon_0 - \hat{\epsilon}(x_t, t)  ||_2^2 \right]
\end{equation}


Another simplification is usually taken and $p_{\theta}(\bm{x}_{t-1} | \bm{x}_t)$ only approximates the means $\bm{\mu}_{\theta}$ and not the variances. For small enough timesteps, the means determine the transitional distributions much stronger than the variances. The network is furhter usually trained to not predict the means directly, but the noise and the means are then determined through a reparameterization.~\autocite{ho2020denoising,nichol2021improved}

\section{Guided Diffusion}
\subsection{Classifier Guidance}
Classifier guidance as termed by Nichol et al. introduces a data consistency term $p(s|x_t)$ in the form of a classifier trained on noisy images, where $s$ is the random variable expressing if an image belongs to a certain class.~\autocite{dhariwal2021diffusion,sohldickstein2015deep} Conditioning on a classifier is sucessfully used by taking gradient ascent steps not only in the direction that maximizes the prior $p(x)$ in a DDPM $\nabla_{x_t} \log p(x_t)$, but also the direction of this conditioning term $\nabla_{x_t} \log p(s|x_t)$. In total, this is equal to Eq.~\ref{eq:mapestimation}
\begin{equation}
    x_{t+1} = \underbrace{x_{t} + \nabla_{x_t} \log p(x_t)}_{x'_{t+1}} + \lambda \nabla_{x_t} \log p(s|x_t)
\end{equation}
with $x'_{t+1}$ being the prediction of the reverse diffusion steps before any conditioning and $\lambda$ an arbitrary factor determining the strength of the guidance.

\subsection{Image-Guided Diffusion}
\label{sec:imageguidance}
Knowledge of the forward process makes it possible to inject information from target images into the latent space where they can be fused with prediction. Lugmayr et al. make use of this for the tasks of image inpainting by always substituting the known image areas during the reverse diffusion.~\autocite{lugmayr2022repaint}
\begin{equation}
    x_{t} = \mathcal{M}^{-1}(x_t) + \mathcal{M}(s_t)
\end{equation}
They further enhance their approach using a resampling strategy, that gives the model more time to harmonize the semantics of the image. An example of such a resampling schedule can be seen in Fig.~\ref{fig:stepsplot}.

Choi et al. also substitute parts of the image in order to guide the inverse diffusion process, but they substitute low-frequency information by using linear filters.~\autocite{choi2021ilvr}
\begin{equation}
    x_{t} = \phi(s_{t}) + (I - \phi) (x_{t})
\end{equation}
They demonstrate strong performance in image translation tasks, e.g. from painting to photo-realistic image.
%% ----------------------------------------------------------------------------
% BIWI SA/MA thesis template
%
% Created 09/29/2006 by Andreas Ess
% Extended 13/02/2009 by Jan Lesniak - jlesniak@vision.ee.ethz.ch
%% ----------------------------------------------------------------------------
\newpage
\chapter{Materials and Methods}
The objectives of the ``Materials and Methods'' section are the following:
\begin{itemize}
    \item \textit{What are tools and methods you used?} Introduce the environment, in which your work has taken place - this can be a software package, a device or a system description. Make sure sufficiently detailed descriptions of the algorithms and concepts (e.g. math) you used shall be placed here.
    \item \textit{What is your work?} Describe (perhaps in a separate chapter) the key component of your work, e.g. an algorithm or software framework you have developed.
\end{itemize}
\section{Latent Variable Models}
Before getting started it is important to define the terms used in the next sections, since they all stem from Bayesian
statistics. The Bayesian theorem can be written as
\begin{equation}
    p(z|x) = \frac{p(x|z)p(z)}{p(x)}
\end{equation}
where it is implicitly assumed that $p$ is a probability density function over two continuous random variables $x$ and $z$. The formula holds in general, but in generative modeling and machine learning it is usually assumed that the letter $z$ represents a random variable in a latent (unobserved) space from which -- after successful training -- new data can be generated by sampling. This requires that $p(z)$ is a simple distribution from which sampling is easy and that the trained model is capable of mapping values from the latent distribution to the true data distribution.

Using above described ordering, the four terms in this formula use distinct names:
\begin{description}
    \item[$p(x)$] is called the \textit{evidence} or the \textit{marginal likelihood}. It encompasses the actual observations of the data.
    \item[$p(z)$] is called the \textit{prior}, since it exposes information on $z$ before any conditioning.
    \item[$p(z|x)$] is called the \textit{posterior}. It describes the distribution over $z$ after (\textit{post}) having seen the evidence $x$.
    \item[$p(x|z)$] is called the \textit{likelihood}. It gives the literal likelihood of observing an example $x$ when choosing the latent space to be a specific $z$.
\end{description}

\section{Variational Autoencoders}
One of the most straightforward examples of a generative model, where the goal is to find such a latent space representation of the training sample distribution, is the Variational Autoencoder (VAE)~\autocite{kingma2022autoencoding}. The name of the VAE stems from the Autoencoder, a network that tries to recreate its output through a bottleneck and thereby learns a compressed representation of the data.~\autocite{https://doi.org/10.1002/aic.690370209} Autoencoders bear similarity to other dimension reduction methods like Principal Component Analysis (PCA) and therefore were first published under the name \textit{Nonlinear principal component analysis}. The \textit{variational} part in the VAE stems from the fact that it does not only learn to recreate input samples through dimensionality reduction, but is also optimized to represent the distribution over the training samples as a combination of a parameterized latent distribution $p_{\theta_z}(z)$ and a neural network mapping $p_{\theta_{NN}}(x|z)$ between the latent space and the sample space, termed decoder. The latent distribution is chosen such that sampling from it is easy (e.g. a multivariate Gaussian, with the parameters being vectors of means and variances). With sufficient dimensionality reduction the encoding should not overfit and the latent space should be a good approximation of the true data manifold. This enables the creation of data, by sampling the latent space and mapping it to the output.

Marginalizing $p_\theta(x)$ requires another approximation of the posterior $p(z|x)$ with a neural network, termed the encoder $p_{\theta_{NN_{in}}}(z|x)$.
\begin{equation}
    p_{\theta}(x) = \int p_{\theta_{NN}}(x|z) p_{\theta_z}(z) dz = \frac{p_{\theta_{NN_{out}}}(x|z) p_{\theta_z}(z)}{p_{\theta_{NN_{in}}}(z|x)}
\end{equation}
A schematic of a VAE, separated into these 3 factors -- encoder, latent distribution and decoder -- is shown in Fig.~\ref{fig:vae}.
\begin{figure}[h]
    \centering
    \includegraphics[width=.5\textwidth]{images/vae.png}
    \caption{VAE schematic: $p(x)$ is approximated through a latent variable model where posterior and likelihood are modeled with neural networks and the prior on the latent variable is modeled through a simple parameterized distribution (often Gaussian). The hope is, that after training, sampling from $p(z)$ and passing it through the neural network $p_{\theta_{NN}}(x|z)$, is the same as sampling from $p(x)$.}
    \label{fig:vae}
\end{figure}

\section{KL Divergence and Variational Lower Bound}
In VAEs, the encoder $p_{\theta}(z|x)$ needs to approximate the posterior $p(z|x)$, therefore a differentiable loss function is needed that compares two probability distributions. One such heavily used measure is the KL (Kullback-Leibler) divergence
\begin{equation}
    KL\left[p_{\theta}(z|x) || p(z|x)\right] = \int \log \frac{p_{\theta}(z|x)}{p(z|x)} p_{\theta}(z|x) dz
\end{equation}
which has the properties of being strictly non-negative and is only 0 if the two distributions are equal.

At the same time, the output of the VAE should fit the true data distribution well, e.g. should maximize the log-likelihood of the evidence $p_{\theta}(x|z)$. This term is mainly responsible for the reconstruction of truthful samples from the latent space. Combining the two terms and inverting the signs (for minimization rather than maximization) gives a loss function known as ELBO (evidence lower bound) or VLB (variational lower bound).
\begin{equation}
    \label{eq:elbo}
    \mathcal{L}_{VLB} = - \log p_{\theta}(x|z) + KL\left[p_{\theta}(z|x) || p(z|x)\right]
\end{equation}

\section{Diffusion Denoising Probabilistic Models}
Diffusion Denoising Probabilistic Models (DDPMs or Diffusion Models) are a generative model that learn the distribution of images in a training set. During training, sample images are gradually destroyed by adding noise over many iterations and a neural network is trained, such that these steps can be inverted.

As the name suggests, image content is diffused in timesteps, therefore we use the random variable $\bm{x}_0$ to represent our original training images, $\bm{x}_t$ for (partially noisy) images at an intermediate timestep and $\bm{x}_T$ for images at the end of the process where all information has been destroyed and the distribution $q(\bm{x}_T)$ largely follows an isotropic Gaussian distribution.

The goal is to train a network that creates a less noisy image $\bm{x}_{t-1}$ from $\bm{x}_t$. If this is achieved we should be able to sample some new $\bm{x}_T$ and generate new samples from the training distribution $q(\bm{x}_0)$ by passing this noisy image many times through the network until the noise is fully removed.

\subsection{Forward Diffusion Process}
\subsubsection{Mathematical Description}
In order to derive a training objective it is important to understand the workings of the \textit{forward diffusion process}. During this process, i.i.d (independent and identically distributed) Gaussian noise is applied to the image over many discrete timesteps. A \textit{variance schedule} defines the means and variances ($\sqrt{1-\beta}$ and $\beta$) of the added noise at every timestep.~\autocite{ho2020denoising} The whole process can be expressed as a Markov chain (depicted in Fig.~\ref{fig:forward_diffusion}), with the factorization
\begin{equation}
    \label{eq:forwardprocess}
    q(\bm{x}_T|\bm{x}_0) = q(\bm{x}_0) \prod_{t=1}^{T} q(\bm{x}_{t}|\bm{x}_{t-1})
\end{equation}
where the transition distributions $q(\bm{x}_t|\bm{x}_{t-1}) = \mathcal{N}(\sqrt{1-\beta_t} \bm{x}_{t-1}, \beta_t I)$. An example of iterative destruction of an image by this process is shown in Fig.~\ref{fig:forward_naoshima}.

\begin{figure}[h]
    \centering
    \includegraphics[width=.5\textwidth]{images/forward_diffusion.png}
    \caption{Forward Diffusion Process: An image is iteratively destroyed by adding normally distributed noise,
        according to a schedule. This represents a Markov process with the transition probability $q(\bm{x}_t|\bm{x}_{t-1})$.}
    \label{fig:forward_diffusion}
\end{figure}

\begin{figure}[h]
    \centering
    \includegraphics[width=\textwidth]{images/forward_naoshima.png}
    \caption{Example of Iterative Image Destruction through Forward Diffusion Process:
        The indices give the time step in the iterative destruction process, where $\beta$ was created according to a linear noise variance schedule (5000 steps from in the 0.001 to 0.02 range and picture resolution of 4016 by 6016 pixels).}
    \label{fig:forward_naoshima}
\end{figure}

Gladly it is not necessary to sample noise again and again in order to arrive at $\bm{x}_t$, since Ho et al. derived a closed-form solution to the sampling procedure.~\autocite{ho2020denoising} For this, the variance schedule is first reparameterized as $1-\beta = \alpha$
\begin{equation}
    q(\bm{x}_t | \bm{x}_{t-1}) = \mathcal{N}(\sqrt{\alpha_t} \bm{x}_{t-1}, (1-\alpha_t) \bm{I})
    \label{eq:forward_alpha}
\end{equation}
and the closed-form solution for $q(\bm{x}_t|\bm{x}_0)$ is derived by introducing the cumulative product $\bar{\alpha_t} = \prod_{s=1}^{t}\alpha_s$ as
\begin{equation}
    q(\bm{x}_t|\bm{x}_0) = \mathcal{N}(\sqrt{\bar{\alpha_t}}\bm{x}_0, (1-\bar{\alpha_t})\bm{I})
    \label{eq:forward_alphadash}
\end{equation}
A choice of $\bar{\alpha_t} \in [0,1]$ in above parameterizaiton ensures that the variance does not explode in the process, but that the SNR (signal-to-noise-ratio) still goes to 0 by gradually attenuating the means, corresponding to the original image. Thanks to the reparameterization with $\bar{\alpha_t}$, the forward process is also not restricted anymore to discrete timesteps, but a continuous schedule can be used.~\autocite{kingma2023variational,song2021scorebased}

The derivation that leads from Eq.~\ref{eq:forward_alpha} to Eq.~\ref{eq:forward_alphadash} is left to appendix~\ref{app:forward}.

\subsubsection{Influence of Scheduling Functions}
The process of information destruction is dependent on the chosen variance schedule, the number of steps and the image size. Beyond the most simple case -- a constant variance over time -- Ho et al. opted for the second most simple option, a linear schedule, where the variance $\beta_t$ grows linearly in $t$.~\autocite{ho2020denoising} Nichol et al. later found that a cosine-based schedule gives better results on lower resolution images, since it does not destruct information quite as quickly, making it more informative in the last few timesteps. They also mention that their cosine schedule is purely based on intuition and they similar functions to perform equally well.~\autocite{nichol2021improved}  Own experiments exploring above mentioned parameters are explained in~\ref{sec:forward_diff_experiments} and plots of the two different variance schedules are visible in Fig.~\ref{fig:alphadash}.

\subsection{Reverse Diffusion Process}
DDPMs can be viewed as latent space models in a similar way that Generative Adversarial Nets or Variational Autoencoders can.~\autocite{goodfellow2014generative,kingma2022autoencoding}

In DDPMs the reverse process is again a Markov chain and can therefore again be factorized as
\begin{equation}
    \label{eq:reverseprocess}
    q(\bm{x}_0|\bm{x}_T) = q(\bm{x}_T) \prod_{t=T}^{1} q(\bm{x}_{t-1}|\bm{x}_{t})
\end{equation}
which means that our network does not learn to approximate the full inversion, but rather just the transition probabilities $q(\bm{x}_{t-1}|\bm{x}_{t})$ in the chain, which are transitions between several intermediate latent distributions. Sohl-Dickstein et al. further showed that the reverse transitions are also Gaussian in the limit of $t \rightarrow 0$, e.g. as long as the diffusion steps are small enough. We therefore approximate
\begin{equation}
    \label{eq:reverseapprox}
    q(\bm{x}_{t-1} | \bm{x}_t) \approx p_{\theta}(\bm{x}_{t-1} | \bm{x}_t) = \mathcal{N}(\bm{\mu}_{\theta}(\bm{x}_t, t),\bm{\Sigma}_{\theta}(\bm{x}_t, t)).
\end{equation}

\subsection{Loss Functions}
The combination of forward $q(\bm{x}_T|\bm{x}_0)$ and reverse process $q(\bm{x}_0|\bm{x}_T)$ can be viewed as a chain of many VAEs and we can again formulate a variational lower bound objective (Eq.~\ref{eq:elbo}) that maximizes log-likelihood of the output and matches transition probabilities.~\autocite{nichol2021improved}
\begin{align}
    \mathcal{L}_0       & = - \log p_{\theta}(x_0|x_1)                                    \\
    \mathcal{L}_{1:T-1} & = KL\left[q(x_{t-1}|x_t, x_0) || p_{\theta}(x_{t-1}|x_t)\right] \\
    \mathcal{L}_T       & = KL\left[q(x_T|x_0) || p(x_T)\right]
\end{align}
The exact posterior $q(x_{t-1}|x_t, x_0)$ is tractable for specific samples of the training distribution, therefore $\mathcal{L}_{1:T-1}$ could be calculated, since KL divergence has a closed form solution for two Gaussian distributions. $\mathcal{L}_T$ is independent of $\theta$ and therefore not used for optimization. It should anyway be very close to zero if the parameterization of the forward process is correct and forward diffused samples get close to $\mathcal{N}(0,\bm{I})$. The first term is only used for performance evaluation in terms of log-likelihood, but not in optimization.

Another simplification is usually taken and $p_{\theta}(\bm{x}_{t-1} | \bm{x}_t)$ only approximates the means $\bm{\mu}_{\theta}$ and not the variances. For small enough timesteps, the means determine the transitional distributions much stronger than the variances. The network is furhter usually trained to not predict the means directly, but the noise and the means are then determined through a reparameterization.~\autocite{ho2020denoising,nichol2021improved}

\section{Image Guided Diffusion}
Both, Choi et al. and Lugmayr et al. make use of unconditional DDPMs for image-guided diffusion for the tasks of image translation in the former and in-painting in the latter.~\autocite{choi2021ilvr,lugmayr2022repaint} Similarly, classifier guidance or CLIP-guidance can be used on unconditional and conditional DDPMs to produce samples of a specific class or matching a prompt in the unconditional case or to further trade off sample variability for sample fidelity.~\autocite{dhariwal2021diffusion} We show here that both approaches can be interpreted as special cases of MAP (maximum a posteriori) estimation and that they can be generalized to other means of guidance during the reverse process.
\subsection{MAP Estimation for Inverse Problems}
MAP estimation is a statistical concept that is often used for optimizing the parameters $\theta$ of a parameterized distribution to observed data $z\sim p(z)$ following Bayes rule
\begin{equation}
    \hat{\theta}_{MAP} = \argmax_{\theta} p(\theta | z) = \argmax_{\theta}\frac{p(z|\theta)p(\theta)}{p(z)}
\end{equation}
or for inverse problems
\begin{equation}
    \hat{x}_{MAP} = \argmax_x p(x|s) = \argmax_x \frac{p(s|x)p(x)}{p(s)}
\end{equation}
where $s \sim p(s)$ is the evidence that is provided by the measured signal, $p(x)$ is a prior on the desired reconstruction and the likelihood term $p(s|x)$ enforces a data consistency between the measured signal and the true distribution, usually a forward model of the data-corrupting process. Since maximizing $p(x|s)$ is the same as maximizing $\log(x|s)$ and $p(s)$ is independent of $\theta$, we can separate the product into a sum.
\begin{align}
    \hat{x}_{MAP} = \argmax_x log p(x|s) & = \argmax_x \log{\frac{p(s|x)p(x)}{p(s)}} \\
                                         & = \argmax_x \log p(s|x)p(x)               \\
                                         & = \argmax_x \log f(x, s)
\end{align}
Such problems are usually optimized using iterative optimization schemes such as gradient ascent $x_{t+1} = x_{t} + \lambda \nabla_{x} f(x, s)$, with $\lambda$ being the step length. It is usually helpful to separate the data consistency term (likelihood) and regularizer (prior) into individual terms for joint optimization with their respective gradients and weights.
\begin{align}
    \label{eq:mapestimation}
    x_{i+1} = x_{i} + \lambda_1 \nabla_{x_i} \log p(s|x_i) + \lambda_2 \nabla_{x_i} \log p(x_i)
\end{align}

\subsection{DDPMs as Priors}
DDPMs approximate a data distribution over training images $p(x)$ and according to Song et al., they do so by learning to approximate gradients of the distribution and taking a gradient ascent step with every iteration of the reverse diffusion process.~\autocite{song2020generative} With this interpretation the DDPM has the exact same form as Eq.~\ref{eq:mapestimation} without the data consistency term $p(s|x)$.
\begin{equation}
    \label{eq:ddpmiteration}
    x_{t-1} = x_{t} + \nabla_{x_t} \log p(x_t)
\end{equation}

\subsection{Classifier Guidance}
Classifier guidance as termed by Nichol et al. introduces a data consistency term $p(s|x_t)$ in the form of a classifier trained on noisy images, where $s$ is the random variable expressing if an image belongs to a certain class.~\autocite{dhariwal2021diffusion,sohldickstein2015deep} Conditioning on a classifier is sucessfully used by taking gradient ascent steps not only in the direction that maximizes the prior $p(x)$ in a DDPM $\nabla_{x_t} \log p(x_t)$, but also the direction of this conditioning term $\nabla_{x_t} \log p(s|x_t)$. In total, this is equal to Eq.~\ref{eq:mapestimation}
\begin{equation}
    x_{t+1} = \underbrace{x_{t} + \nabla_{x_t} \log p(x_t)}_{x'_{t+1}} + \lambda \nabla_{x_t} \log p(s|x_t)
\end{equation}
with $x'_{t+1}$ being the prediction of the reverse diffusion steps before any conditioning and $\lambda$ an arbitrary factor determining the strength of the guidance.

\subsection{Gradients and Closed-Forms of Data Consistency Terms}
Starting from Eq.~\ref{eq:ddpmiteration}, the data consistency term can be reintroduced into the DDPM model.

Choi et al. guide the diffusion process by trying to match the low frequency content of a target image to the low frequency content of the prediction $\argmin_x \phi(s) - phi(x)$. They do this by using a very simple linear data consistency function, corresponding to a difference between linear low-pass filtered representations, which they call ILVR (iterative latent variable refinement)
\begin{align}
    \label{eq:ilvr}
    x_{t-1} & = \phi(s_{t}) + (I - \phi) (x_{t})  \\
    x_{t-1} & = x_{t} + \phi(s_{t}) - \phi(x_{t}) \\
    x_{t-1} & = x_{t} + \phi(s_{t} -x_{t})
\end{align}
where $\phi$ is a linear filter operation and $s_t$ is obtained by using the forward process on the target image.~\autocite{choi2021ilvr}

Similarly, Lugmayr et al. use a conditioning on known parts of the image for the inpainting operation, which effectively boils down to applying a linear mask that zeroes out known parts in the prediction and replaces them with outputs from the forward process.
\begin{align}
    \label{eq:repaint}
    % Mx_t (0 at unknown, 1 at known) removes known parts from x_t and leaves unknown
    % Ms_t (0 at unknown, 1 at known) leaves prediction in unknown region and replaces known region
    x_{t-1} & = x_t - \mathcal{M}(x_t) + \mathcal{M}(s_t) \\
    x_{t-1} & = x_t + \mathcal{M}(s_t - x_t)
\end{align}
As is easily seen, Eq.~\ref{eq:ilvr} and Eq.~\ref{eq:repaint} take the same form and can be interpretated as simultaneously taking gradient ascent steps for optimizing $\argmax_x \log p(x)$ and $\argmax_x \log p(s|x)$.

Assuming distribution over latent predictions and targets at diffusion timestep $t$, $p(s|t)$ and $p(x|t)$, and $\mathcal{L}$ corresponding to an arbitrary linear operation
\begin{align}
    \mathcal{L}(p(s|t) - p(x|t))                                                       & = \nabla_{x_t} \log p(s_t|x_t)         \\
    \int \mathcal{L}(p(s|t) - p(x|t)) dt                                               & = \int \nabla_{x_t} \log p(s_t|x_t) dt \\
    \int \mathcal{L}(p(s|t))dt - \int \mathcal{L}(p(x|t))dt                            & = \log p(s|x)                          \\
    \mathcal{L} \left(\int p(s|t)dt \right) - \mathcal{L} \left( \int p(x|t)dt \right) & = \log p(s|x)                          \\
    \mathcal{L}(p(s)) - \mathcal{L}(p(x))                                              & = \log p(s|x)
\end{align}
where the left side of the equation is the original
%% ----------------------------------------------------------------------------
% BIWI SA/MA thesis template
%
% Created 09/29/2006 by Andreas Ess
% Extended 13/02/2009 by Jan Lesniak - jlesniak@vision.ee.ethz.ch
%% ----------------------------------------------------------------------------
\newpage
\chapter{Experiments and Results}
Describe the evaluation you did in a way, such that an independent researcher can repeat it. Cover the following questions:
\begin{itemize}
    \item \textit{What is the experimental setup and methodology?} Describe the setting of the experiments and give all the parameters in detail which you have used. Give a detailed account of how the experiment was conducted.
    \item \textit{What are your results?} In this section, a \emph{clear description} of the results is given. If you produced lots of data, include only representative data here and put all results into the appendix.
\end{itemize}

\section{Influence of Schedules and Image Size on the Forward Diffusion}
\label{sec:forward_diff_experiments}
Ho et al. had derived a closed form solution to the forward process of DDPMs and Nichol et al. investigated alternative options for the noise scheduling.~\autocite{ho2020denoising,nichol2021improved} They concluded that the important parameters to model are not the variances $\beta$ of the transitions, but the variances $1-\bar{\alpha}$ of the closed-form forward process, since they are the ones responsible for the destruction of information.

They decided to go with a squared cosine function, since this would be close to linear smooth out towards the critical beginning and end points of the process. In Fig.\ref{fig:alphadash} you can see how $1-\bar{\alpha}$ and $\beta$ behave for both approaches. It is immediately visible that the variances reach the maximum too early and flatten out for the linear schedule. This leads to the intuition that the last few steps are not very useful.

\begin{figure}[h]
    \centering
    \includegraphics[width=.7\textwidth]{images/variance_schedule_alphadash.png}
    \caption{Variance Schedule Approaches: Modeling the $1-\bar{\alpha}$ as an approximate linear function (right cosine) and deriving $\beta$ (left cosine), or modeling $\beta$ as a linear function (left linear) and deriving $1-\bar{\alpha}$.}
    \label{fig:alphadash}
\end{figure}

The intution can experimentally confirmed by measuring how closely we get to isotropic noise when passing samples through the forward process. For this a batch of 50 times the same image was passed through the different steps of the process and the covariance matrix was calculated. As a metric for how close the covariance matrix was to the identity covariance matrix of pure i.i.d Gaussian noise, the identity matrix was subtracted and the mean of the absolute value of the matrix calculated. The results can be seen in Fig.~\ref{fig:noisecloseness} and confirm the intuition: When using linear scheduling we reach the closest point to pure noise already after around 600 steps for small images, and after around 700 for larger images. Cosine scheduling also performs worse on smaller images than on larger ones, but is still capable providing value for at least 850 timesteps.

\begin{figure}[h]
    \centering
    \includegraphics[width=.7\textwidth]{images/frobenius_norm.png}
    \caption{Closeness to noise for linear scheduling (left) and cosine scheduling (right).}
    \label{fig:noisecloseness}
\end{figure}
%% ----------------------------------------------------------------------------
% BIWI SA/MA thesis template
%
% Created 09/29/2006 by Andreas Ess
% Extended 13/02/2009 by Jan Lesniak - jlesniak@vision.ee.ethz.ch
%% ----------------------------------------------------------------------------
\newpage
\chapter{Discussion and Conclusion}
The focus of this work was on understanding DDPMs and conditioning them for the task of reconstructing undersampled MRI. The modeling and loss functions of the DDPM were derived in great detail and a package was created that implements DDPMs from scratch and provides the necessary utilities for efficient training, logging and sampling. Prior work by Lugmayr et al. and Choi et al.~\autocite{lugmayr2022repaint,choi2021ilvr} had introduced a conditioning method for inpainting and image-to-image translation respectively that used the known posterior of the forward process to substitute predicted information with known information in the latent space. Both their works, RePaint and ILVR, were successfully implemented with a model trained on MRI data and resampling for RePaint was equally observed to improve semantics of the reconstruction. Lugmayr et al.'s work was subsequently adapted to the task of MRI reconstruction, but the direct adaptation was shown to produce insufficient reconstructions. In order to avoid image artifacts, like aliasing and ringing, Choi et al.'s filtering was reintroduced in the form of a scheduled Gaussian filter that only conditions the model on low frequencies early in the reverse diffusion process and introduces higher frequencies later. The intuition behind this approach was that low frequencies would carry very little information for high noise variances of the latent space, since this is a general property of natural images. While reconstruction quality failed to improve through the use of filtering, the analysis of outcome variability offered a unique view on the hierarchy of features in a DDPM. As hypothesized, global image features corresponding to low frequencies were determined early in the DDPM, whereas high frequency showed variety until very late in the denoising process.

Using the score-based interpretation of the DDPM and adapting classifier guidance to accomodate other types of data consistency functions proved to be a much more flexible and powerful approach to the conditioning problem. \textit{Loss guidance} performed very well out of the box for the lower accelerations in the range $\approx 3-6$, with high perceptual sample quality and very good MSE scores. For the highest acceleration of $>11$, direct sampling often produced aliasing artifacts, indicating that the final prediction did not correspond to the same distributional mode as the samples used for guidance, which led to frequency mismatch. Using the \textit{long-grained} resampling technique, this issue was resolved and the very high acceleration of $>11$ produced reconstructions that almost reached the quality of the ones with direct sampling and acceleration $\approx 5.5$.

Loss guidance proved to be a powerful tool for including DDPMs into iterative reconstruction schemes and resampling allowed for the increase of iteration steps, that was helpful for the most challenging reconstruction tasks. Since the focus of this work was on finding appropriate conditioning methods, using a set of dedicated test images was not the first priority, but future work should investigate how well this method generalizes to unseen images. The uncertainty of the predictions was also not investigated in this work and since loss guidance coincides with MAP estimation, it might be interesting to introduce additional regularizers that can lower the uncertainty. Finally, models capable of higher resolution should be trained and evaluated. The highest resolution used in this work was $128\times 128$ and section~\ref{sec:networkarch} already presented ideas how fully convolutional architectures could be trained to generalize on higher image resolutions.
\input{chap/conclusion.tex}

\printbibliography[title=References]


%% ----------------------------------------------------------------------------
% If Appendix is needed
%% ----------------------------------------------------------------------------
\appendix
%% ----------------------------------------------------------------------------
% BIWI SA/MA thesis template
%
% Created 09/29/2006 by Andreas Ess
% Extended 13/02/2009 by Jan Lesniak - jlesniak@vision.ee.ethz.ch
%% ----------------------------------------------------------------------------
\chapter{Extended Derivations}
\section{Forward Process Marginal}
\label{app:forward}
Starting with transition distributions
\begin{equation}
    q(\bm{x}_t|\bm{x}_{t-1}) = \mathcal{N}(\sqrt{1-\beta_t} \bm{x}_{t-1}, \beta_t \bm{I})
\end{equation}
the reparameterization $\alpha = 1 - \beta$ is introduced
\begin{equation}
    q(\bm{x}_t|\bm{x}_{t-1}) = \mathcal{N}(\sqrt{\alpha_t} \bm{x}_{t-1}, (1-\alpha_t) \bm{I})
\end{equation}
which can be reformulated using the reparameterization trick as
\begin{align}
    \bm{x}_t & = \sqrt{\alpha_t}\bm{x}_{t-1} + \sqrt{1-\alpha_t}\cdot\mathcal{N}(\bm{0}, \bm{I}) \
             & = \sqrt{\alpha_t}\bm{x}_{t-1} + \sqrt{1-\alpha_t} \cdot \bm{\epsilon}
\end{align}
with $\bm{\epsilon} \sim \mathcal{N}(\bm{0}, \bm{I})$. For the derivation it is helpful to use proper indices on the noise variables $\bm{\epsilon}_t$ and track them
\begin{equation}
    \bm{x}_{t} = \sqrt{\alpha_t}\bm{x}_{t-1} + \sqrt{1-\alpha_t}\bm{\epsilon_{t-1}}.
    \label{eq:forward_randomvar}
\end{equation}
The next term $\bm{x}_{t-1}$ can now be insterted into the formula by again using the reparameterization trick. Recalling that the sum $Z = X + Y$ of two normally distributed random variables $X \sim \mathcal{N}(\mu_X, \sigma_Y^2)$ and $Y \sim \mathcal{N}(\mu_Y, \sigma_Y^2)$ is again normally distributed according to $Z \sim \mathcal{N}(\mu_X + \mu_Y, \sigma_X^2 + \sigma_Y^2)$
\begin{align}
    x_t & = \sqrt{\alpha_t} \left( \sqrt{\alpha_{t-1}} \bm{x}_{t-2} + \sqrt{1-\alpha_{t-1}}\bm{\epsilon}_{t-2} \right) + \sqrt{1-\alpha_{t}} \bm{\epsilon}_{t-1} \\
        & = \sqrt{\alpha_{t}\alpha_{t-1}} \bm{x}_{t-2} + \sqrt{\alpha_{t}(1-\alpha_{t-1})} \bm{\epsilon}_{t-2} + \sqrt{1-\alpha_{t}} \bm{\epsilon}_{t-1}         \\
        & = \sqrt{\alpha_{t}\alpha_{t-1}} \bm{x}_{t-2} + \sqrt{\alpha_{t}(1-\alpha_{t-1}) + (1-\alpha_{t})} \bm{\bar{\epsilon}}_{t-2}
\end{align}
where $\bm{\bar{\epsilon}}_{t-2}$ is the noise variable for the sum of the random random variables up to $t-2$ (again $\bm{\bar{\epsilon}}_{t-2} \sim \mathcal{N}(\bm{0}, \bm{I})$). The second term can be simplified to
\begin{equation}
    \bm{x}_t = \sqrt{\alpha_{t}\alpha_{t-1}} \bm{x}_{t-2} + \sqrt{1-\alpha_t\alpha_{t-1}} \bm{\bar{\epsilon}}_{t-2}
\end{equation}
which is exactly the same form as in Eq.~\ref{eq:forward_randomvar}. The same procedure can be repeated in a recursive manner until the arrival at
\begin{equation}
    \bm{x}_t = \sqrt{\prod_{s=1}^{t}\alpha_s} \bm{x}_{0} + \sqrt{1-\prod_{s=1}^{t}\alpha_s} \bm{\bar{\epsilon}}_{0}
\end{equation}
At which point we define $\bar{\alpha_t} = \prod_{s=1}^{t}\alpha_s$ and arrive at the final forms
\begin{align}
    \bm{x}_t                           & = \sqrt{\bar{\alpha}_{t}} \bm{x}_{0} + \sqrt{1-\bar{\alpha}_{t}} \bm{\bar{\epsilon}}_{0} \\
    \Rightarrow q(\bm{x}_t|\bm{x}_{0}) & = \mathcal{N}(\sqrt{\bar{\alpha}_t} \bm{x}_{0}, (1-\bar{\alpha}) \bm{I})
\end{align}
with  as before.

\section{Derivation of Reverse Process Parameterization}
\begin{equation}
    q(\bm{x}_t|\bm{x}_{0}) = \frac{q(\bm{x}_t|\bm{x}_{t-1},\bm{x}_{0})q(\bm{x}_{t-1}|\bm{x}_{0})}{q(\bm{x}_t|\bm{x}_{0})}
\end{equation}
where $q(\bm{x}_t|\bm{x}_{t-1},\bm{x}_{0})$ is independent of $\bm{x}_0$ given $\bm{x}_{t-1}$ thanks to the factorization as a Markov chain and therefore
\begin{align}
    q(\bm{x}_t|\bm{x}_{0}) & = \frac{q(\bm{x}_t|\bm{x}_{t-1})q(\bm{x}_{t-1}|\bm{x}_{0})}{q(\bm{x}_t|\bm{x}_{0})}                                                                                                                                                       \\
                           & = \frac{\mathcal{N}(\sqrt{1-\beta_t}\bm{x_{t-1}}, \beta_t \bm{I}) \cdot \mathcal{N}(\sqrt{\bar{\alpha}_{t-1}}\bm{x_{t-1}}, (1-\bar{\alpha}_{t-1}) \bm{I})}{\mathcal{N}(\sqrt{\bar{\alpha}_{t}}\bm{x_{t-1}}, (1-\bar{\alpha}_{t}) \bm{I})}
\end{align}
The formula for the multivariate Gaussian distribution simplifies as follows for the case of a diagonal covariance matrix.
\begin{align}
\end{align}
\begin{align}
    q(\bm{x}_t|\bm{x}_{0}) & \propto \exp \left( -\left(\frac{\left(\bm{x}_{t}-\sqrt{\alpha_{t}} \bm{x}_{t-1}\right)^{2}}{2\left(1-\alpha_{t}\right)}+\frac{\left(\bm{x}_{t-1}-\sqrt{\bar{\alpha}_{t-1}} \bm{x}_{0}\right)^{2}}{2\left(1-\bar{\alpha}_{t-1}\right)}-\frac{\left(\bm{x}_{t}-\sqrt{\bar{\alpha}_{t}} \bm{x}_{0}\right)^{2}}{2\left(1-\bar{\alpha}_{t}\right)}\right) \right)                                                     \\
                           & =\exp \left(-\frac{1}{2}\left(\frac{\left(\bm{x}_{t}-\sqrt{\alpha_{t}} \bm{x}_{t-1}\right)^{2}}{1-\alpha_{t}}+\frac{\left(\bm{x}_{t-1}-\sqrt{\bar{\alpha}_{t-1}} \bm{x}_{0}\right)^{2}}{1-\bar{\alpha}_{t-1}}-\frac{\left(\bm{x}_{t}-\sqrt{\bar{\alpha}_{t}} \bm{x}_{0}\right)^{2}}{1-\bar{\alpha}_{t}}\right)\right)                                                                                             \\
                           & =\exp \left(-\frac{1}{2}\left(\frac{\left(-2 \sqrt{\alpha_{t}} \bm{x}_{t} \bm{x}_{t-1}+\alpha_{t} \bm{x}_{t-1}^{2}\right)}{1-\alpha_{t}}+\frac{\left(\bm{x}_{t-1}^{2}-2 \sqrt{\bar{\alpha}_{t-1}} \bm{x}_{t-1} \bm{x}_{0}\right)}{1-\bar{\alpha}_{t-1}}+C\left(\bm{x}_{t}, \bm{x}_{0}\right)\right)\right)                                                                                                        \\
                           & \propto \exp \left(-\frac{1}{2}\left(-\frac{2 \sqrt{\alpha_{t}} \bm{x}_{t} \bm{x}_{t-1}}{1-\alpha_{t}}+\frac{\alpha_{t} \bm{x}_{t-1}^{2}}{1-\alpha_{t}}+\frac{\bm{x}_{t-1}^{2}}{1-\bar{\alpha}_{t-1}}-\frac{2 \sqrt{\bar{\alpha}_{t-1}} \bm{x}_{t-1} \bm{x}_{0}}{1-\bar{\alpha}_{t-1}}\right)\right)                                                                                                              \\
                           & =\exp \left(-\frac{1}{2}\left(\left(\frac{\alpha_{t}}{1-\alpha_{t}}+\frac{1}{1-\bar{\alpha}_{t-1}}\right) \bm{x}_{t-1}^{2}-2\left(\frac{\sqrt{\alpha_{t}} \bm{x}_{t}}{1-\alpha_{t}}+\frac{\sqrt{\bar{\alpha}_{t-1}} \bm{x}_{0}}{1-\bar{\alpha}_{t-1}}\right) \bm{x}_{t-1}\right)\right)                                                                                                                           \\
                           & =\exp \left(-\frac{1}{2}\left(\frac{\alpha_{t}\left(1-\bar{\alpha}_{t-1}\right)+1-\alpha_{t}}{\left(1-\alpha_{t}\right)\left(1-\bar{\alpha}_{t-1}\right)} \bm{x}_{t-1}^{2}-2\left(\frac{\sqrt{\alpha_{t}} \bm{x}_{t}}{1-\alpha_{t}}+\frac{\sqrt{\alpha_{t-1}} \bm{x}_{0}}{1-\bar{\alpha}_{t-1}}\right) \bm{x}_{t-1}\right)\right)                                                                                 \\
                           & =\exp \left(-\frac{1}{2}\left(\frac{\alpha_{t}-\bar{\alpha}_{t}+1-\alpha_{t}}{\left(1-\alpha_{t}\right)\left(1-\bar{\alpha}_{t-1}\right)} \bm{x}_{t-1}^{2}-2\left(\frac{\sqrt{\alpha_{t}} \bm{x}_{t}}{1-\alpha_{t}}+\frac{\sqrt{\alpha_{t-1}} \bm{x}_{0}}{1-\bar{\alpha}_{t-1}}\right) \bm{x}_{t-1}\right)\right)                                                                                                 \\
                           & =\exp \left(-\frac{1}{2}\left(\frac{1-\bar{\alpha}_{t}}{\left(1-\alpha_{t}\right)\left(1-\bar{\alpha}_{t-1}\right)} \bm{x}_{t-1}^{2}-2\left(\frac{\sqrt{\alpha_{t}} \bm{x}_{t}}{1-\alpha_{t}}+\frac{\sqrt{\alpha_{t-1}} \bm{x}_{0}}{1-\bar{\alpha}_{t-1}}\right) \bm{x}_{t-1}\right)\right)                                                                                                                       \\
                           & =\exp \left(-\frac{1}{2}\left(\frac{1-\bar{\alpha}_{t}}{\left(1-\alpha_{t}\right)\left(1-\bar{\alpha}_{t-1}\right)}\right)\left(\bm{x}_{t-1}^{2}-2 \frac{\left(\frac{\sqrt{\alpha_{\alpha}} \bm{x}_{t}}{1-\alpha_{t}}+\frac{\sqrt{\bar{\alpha}_{t-1}} \bm{x}_{0}}{1-\bar{\alpha}_{t-1}}\right)}{\frac{1-\bar{\alpha}_{t}}{\left(1-\alpha_{t}\right)\left(1-\bar{\alpha}_{t-1}\right)}} \bm{x}_{t-1}\right)\right) \\
                           & =\exp \left(-\frac{1}{2}\left(\frac{1-\bar{\alpha}_{t}}{\left(1-\alpha_{t}\right)\left(1-\bar{\alpha}_{t-1}\right)}\right)\left(\bm{x}_{t-1}^{2}-2 \frac{\left(\frac{\sqrt{\alpha_{t}} \bm{x}_{t}}{1-\alpha_{t}}+\frac{\sqrt{\bar{\alpha}_{t-1}} \bm{x}_{0}}{1-\bar{\alpha}_{t-1}}\right)\left(1-\alpha_{t}\right)\left(1-\bar{\alpha}_{t-1}\right)}{1-\bar{\alpha}_{t}} \bm{x}_{t-1}\right)\right)               \\
                           & =\exp \left(-\frac{1}{2}\left(\frac{1}{\frac{\left(1-\alpha_{t}\right)\left(1-\bar{\alpha}_{t-1}\right)}{1-\bar{\alpha}_{t}}}\right)\left(\bm{x}_{t-1}^{2}-2 \frac{\sqrt{\alpha_{t}}\left(1-\bar{\alpha}_{t-1}\right) \bm{x}_{t}+\sqrt{\bar{\alpha}_{t-1}}\left(1-\alpha_{t}\right) \bm{x}_{0}}{1-\bar{\alpha}_{t}} \bm{x}_{t-1}\right)\right)                                                                    \\
                           & \propto \frac{\mathcal{N}\left(\bm{x}_{t-1} ; \frac{\sqrt{\alpha_{t}}\left(1-\bar{\alpha}_{t-1}\right) \bm{x}_{t}+\sqrt{\bar{\alpha}_{t-1}}\left(1-\alpha_{t}\right) \bm{x}_{0}}{1-\bar{\alpha}_{t}}, \frac{\left(1-\alpha_{t}\right)\left(1-\bar{\alpha}_{t-1}\right)}{1-\bar{\alpha}_{t}} \mathbf{I}\right)}{\mu_{q}\left(\bm{x}_{t,}, \bm{x}_{0}\right)}
\end{align}

\section{Derivation of ELBO/VLB}
In the case of a VAE we have
\begin{align}
    \log p_{\theta}(x) & = \log p_{\theta}(x) \int p_{\theta_{NN}}(z|x)dz                                                                                                                                                                           \\
                       & = \int \log p_{\theta}(x) p_{\theta_{NN}}(z|x)dz                                                                                                                                                                           \\
                       & = \mathbb{E}_{z\sim p_{\theta_{NN}}(z|x)}\left[\log p_{\theta}(x) \right]                                                                                                                                                  \\
                       & = \mathbb{E}_{z\sim p_{\theta_{NN}}(z|x)}\left[\log \frac{p_{\theta_{NN}}(x|z)p_{\theta_z}(z)}{p(z|x)}\right]   \label{eq:A31}                                                                                             \\
                       & = \mathbb{E}_{z\sim p_{\theta_{NN}}(z|x)}\left[\log \frac{p_{\theta_{NN}}(x|z)p_{\theta_z}(z)p_{\theta_{NN}}(z|x)}{p(z|x)p_{\theta_{NN}}(z|x)}\right]                                                                      \\
                       & = \mathbb{E}_{z\sim p_{\theta_{NN}}(z|x)}\left[\log \frac{p_{\theta_{NN}}(x|z)p_{\theta_z}(z)}{p_{\theta_{NN}}(z|x)}\right] + \mathbb{E}_{z\sim p_{\theta_{NN}}(z|x)}\left[\log \frac{p_{\theta_{NN}}(z|x)}{p(z|x)}\right] \\
                       & = \mathbb{E}_{z\sim p_{\theta_{NN}}(z|x)}\left[\log \frac{p_{\theta_{NN}}(x|z)p_{\theta_z}(z)}{p_{\theta_{NN}}(z|x)}\right] + KL \left[p_{\theta_{NN}}(z|x)||p(z|x)\right].
\end{align}
Realize that only if $p_{\theta_{NN}}(z|x) = p(z|x)$ -- which is exactly when the the KL divergence is 0 -- we would get our original marginal log-likelihood $p_{\theta}(x)$ from the first term, as defined in Eq.~\ref{eq:likelihoodvae}, by substituting and calculating back from Eq.~\ref{eq:A31}.
\begin{align}
    \mathbb{E}_{z\sim p_{\theta_{NN}}(z|x)}\left[\log \frac{p_{\theta_{NN}}(x|z)p_{\theta_z}(z)}{p_{\theta_{NN}}(z|x)}\right] & \stackrel{p_{\theta_{NN}}(z|x) = p(z|x)}{=} \mathbb{E}_{z\sim p_{\theta_{NN}}(z|x)}\underbrace{\left[\log \frac{p_{\theta_{NN}}(x|z)p_{\theta_z}(z)}{p(z|x)}\right]}_{p_{\theta}(x)} \\
                                                                                                                              & = \int \log p_{\theta}(x) p_{\theta_{NN}}(z|x) dz                                                                                                                                    \\
                                                                                                                              & = \log p_{\theta}(x)
\end{align}

The derivation for the DDPM is similar
\begin{align}
    dummy & = x \\
    blubb & = d
\end{align}
\chapter{Training Metrics}
\begin{table}
    \centering
    \caption[Hyperparameter Overview dutifulpond10]{Overview over the Hyperparameters of dutifulpond10: The model was trained on all RSS reconstructions of the fastMRI brain dataset at a resolution of $128\times 128$, at a batch size of 48 and using the Adam optimizer at an initial learning rate of 0.0001.}
    \begin{tabular}{l l}
        Hyperparameter       & Value                             \\
        \hline
        dataset              & utils.datasets.FastMRIBrainTrain  \\
        dropout              & 0                                 \\
        backbone             & models.unet.UNet                  \\
        img\_size            & 128                               \\
        attention            & False                             \\
        loss\_func           & torch.nn.functional.mse\_loss     \\
        optimizer            & torch.optim.adam.Adam             \\
        activation           & torch.nn.modules.activation.SiLU  \\
        batch\_size          & 48                                \\
        in\_channels         & 1                                 \\
        kernel\_size         & 3                                 \\
        architecture         & models.diffusion.DiffusionModel   \\
        forward\_diff        & models.diffusion.ForwardDiffusion \\
        time\_enc\_dim       & 512                               \\
        learning\_rate       & 0.0001                            \\
        max\_timesteps       & 1000                              \\
        schedule\_type       & cosine                            \\
        from\_checkpoint     & False                             \\
        mixed\_precision     & True                              \\
        backbone\_enc\_depth & 5                                 \\
        unet\_init\_channels & 64                                \\
    \end{tabular}
\end{table}
\begin{figure}
    \centering
    \includegraphics[width=\textwidth]{images/dutifulpondmetrics.png}
    \caption[Metrics from the Training Process of dutifulpond10]{Metrics from the Training Process of dutifulpond10}
\end{figure}

\chapter{Samples \& Plots}
\begin{figure}[h]
    \centering
    \includegraphics[width=\textwidth]{images/directsampling_comparison.png}
    \caption[Direct Sampling with Varying Masks and Guidance Factors]{Direct Sampling with Varying Masks and Guidance Factors: a) - b) Reconstructions of two samples, reconstructed using different masks (vertical) and different guidance factors (horizontal). For comparison, the top row is always the original samples. Low guidance factors lead to samples with less contrast and less adherence to the original and high guidance factors combined can cause aliasing as observed in the lower right corner of a).}
\end{figure}

\begin{figure}[h]
    \centering
    \includegraphics[width=.66\textwidth]{images/gradientspectra.png}
    \caption[Values and Spectra of Loss Gradients]{Values and Spectra of Loss Gradients}
    \label{fig:lossgradients}
\end{figure}

\begin{figure}[h]
    \centering
    \includegraphics[width=\textwidth]{images/kspacedistribution.png}
    \caption[Noise Distribution of Latent K-Space]{Noise Distribution of Latent Space and Latent K-Space: a) Measured noise variances of forward process for image space as well as absolute, real and imaginary parts of k-space. The variances for the real and imaginary part are the same and were only shifted for visibility. They correspond to half the variances of the image space as can be seen in b) or when comparing the histograms in c), e) and f). The distribution of the absolute values of k-space follows the Rice distribution.}
    \label{fig:kspacedistribution}
\end{figure}

\begin{figure}[h]
    \centering
    \includegraphics[width=\textwidth]{images/stepsplot.png}
    \caption[Stepsplot for RePaint Resampling]{Stepsplot for RePaint Resampling}
    \label{fig:stepsplot}
\end{figure}

%% ----------------------------------------------------------------------------
% Bibliography is stored in references.bib file, and can often be found
% online on webpages like dblp.uni-trier.de
%
% To include it in your thesis, run
%  pdflatex main
%  bibtex main
%  pdflatex main
%  pdflatex main
%
% This ensures all references are done correctly.
%% ----------------------------------------------------------------------------

\end{document}

